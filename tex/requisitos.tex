% Contenidos del capítulo
%Las secciones presentadas son orientativas y no representan necesariamente la organización que debe tener este capítulo.

\section{Requisitos}
Requisitos funcionales y no funcionales del proyecto.

Se debe optar por formular los requisitos de forma que se pueda
conocer si se han alcanzado o no a la finalización del proyecto. Por
ejemplo, es difícil valorar si el siguiente requisito funcional se
alcanza o no: \textit{El sistema debe retornar una respuesta en un tiempo
razonable cuando tenga muchos usuarios concurrentes}. ¿Cuánto es un
tiempo razonable?, ¿cuantos son muchos usuarios?. Sin embargo, si se
formula de este otro modo: \textit{El sistema debe retornar una
respuesta en menos de un segundo cuando tenga 200 usuarios
concurrentes}, es fácil comprobar si se ha alcanzado ejecutando un
plan de pruebas, por ejemplo con JMeter.

\section{Especificaciones}
Especificación del proyecto a partir de los requisitos.

\section{Planificación y estimación de costes}
Describir el tipo de metodología de desarrollo que se va a utilizar
(cascada, ágil, etc).
Tareas a realizar, estimación de la duración de las tareas, y
distribución temporal (por ejemplo con un diagrama de Gantt).

Tareas a realizar, estimación de la duración de las tareas, y
distribución temporal (por ejemplo con un diagrama de Gantt).

Costes de personal (teniendo en cuenta los costes de seguridad
social), de hardware (imputando solo la duración del proyecto y teniendo
en cuenta que los equipos se amortizan en 3 o 4 años) y/o de software.
Además, hay que añadir costes indirectos.

\section{Riesgos}

Identificación de los riesgos que pueden aparecer durante el
desarrollo del proyecto, su probabilidad de ocurrencia, su impacto en
el proyecto y las medidas que se podrían adoptar para mitigarlos.

\section{Viabilidad}
En este apartado, dependiendo de la naturaleza del proyecto, se
debería analizar la viabilidad técnica y la viabilidad económica. Para
la viabilidad técnica hay que analizar si los recursos
necesarios (herramientas, conocimientos, experiencia, etc)  para llevar
a cabo el proyecto permiten realizarlo en el tiempo previsto.
En cuanto a la viabilidad económica hay que evaluar si el proyecto
será rentable cuando esté operativo.