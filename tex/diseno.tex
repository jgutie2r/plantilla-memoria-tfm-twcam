% Contenidos del capítulo.

El capítulo de diseño presenta \textbf{cómo} se va a abordar desde el punto de vista técnico
lo que se ha presentado en la fase de análisis.

Los contenidos  presentados son orientativos y se deberán adaptar a la
naturaleza del trabajo realizado.

Toda memoria de \gls{tfm} debería describir, usando \gls{uml}, los siguientes aspectos:

\begin{itemize}
    \item Una arquitectura general o de referencia, para lo cual se recomienda utilizar un diagrama \gls{uml} de componentes.
    \item El modelo de datos de alto nivel mediante un diagrama \gls{uml} de clases, que se debe completar con el diseño de la base de datos mediante diagramas entidad/relación, modelo lógico o modelo lógico.
    \item Diagramas \gls{uml} de secuencia para describir las principales interacciones entre componentes y clases.
    \item Diagramas \gls{uml} de despliegue para describir como son desplegadas las componentes en los nodos físicos/virtuales durante el desarrollo y las pruebas.
    \item  Un diagrama \gls{uml} de despliegue ideal para describir como serían desplegadas las componentes en los nodos físicos/virtuales cuando la aplicación esté en producción.
    \item Otros aspectos como el diseño de pantallas, etc.
\end{itemize}